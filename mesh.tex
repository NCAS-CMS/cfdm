4.3 Coordinates: dimension coordinate and auxiliary constructs

Coordinate constructs (Fig. 10) provide information which locate the
cells of the domain and which depend on a subset of the domain axis
constructs. As previously discussed, there are two distinct types of
coordinate construct: a dimension coordinate construct provides
monotonic numeric coordinates for a single domain axis, and an
auxiliary coordinate construct provides any type of coordinate
information for one or more of the domain axes.

In both cases, the coordinate construct consists of a data array of
the coordinate values which spans a subset of the domain axis
constructs, an optional array of cell bounds recording the extents of
each cell, and properties to describe the coordinates (in the same
sense as for the field construct). An array of cell bounds spans the
same domain axes as its coordinate array, with the addition of an
extra dimension whose size is that of the number of vertices of each
cell. This extra dimension does not correspond to a domain axis
constructsince it does not relate to an independent axis of the domain
(for example, the bounds dimension defined at line 6 in Fig. 3 does
not correspond to a domain axis construct).

The dimension coordinate construct is able to unambiguously describe
cell locations because a domain axis can be associated with at most
one dimension coordinate construct, whose data array values must all
be non-missing and strictly monotonically increasing or
decreasing. They must also all be of the same numeric data type. If
cell bounds are provided, then each cell must have exactly two
vertices. CF-netCDF coordinate variables and numeric scalar coordinate
variables correspond to dimension coordinate constructs.

Auxiliary coordinate constructs have to be used, instead of dimension
coordinate constructs, when a single domain axis requires more then
one set of coordinate values, when coordinate values are not numeric,
strictly monotonic, or contain missing values, or when they vary along
more than one domain axis construct simultaneously. CF-netCDF
auxiliary coordinate variables and non-numeric scalar coordinate
variables correspond to auxiliary coordinate constructs.

If a domain axis construct does not correspond to a continuous
physical quantity, then it is not necessary for it to be associated
with a dimension coordinate construct. For example, this is the case
for an axis that runs over ocean basins or area types, or for a domain
axis that indexes a time series at scattered points. In such cases,
one-dimensional auxiliary coordinate constructs could be used to store
coordinate values. These axes are discrete axes in CF-netCDF.

%%  SHAPE %%%%%%%%%%%%%%%%%

A coordinate construct may optionally contain one or more ``cell
extent'' properties to describe how the cell bounds relate to cell
extents, i.e. how the vertices of a cell connect to define the edges
of a region of the domain. If present, these properties include a
descriptive string that specifies the type of cell construction in
use; and may include arrays of ancillary values that contain
parameters needed to map each cell's vertices to its extent. Such
arrays span the same domain axes as the coordinate array, with the
addition of an extra dimension whose size is that of the maximum
number of ancillary values required for any cell (padding any unused
elements with missing values). In the absence of these properties,
cell extent's are derived by the standard ruls described in the CF
conventions.

Cell extent properties correspond to particular attributes of a
variety of CF-netCDF variables; which ones depends on the type of
relationship between the cell bounds and the cell extents:

\begin{itemize}

\item Climatological time axis. Each cell represets a disjoint set of
  regions of the domain. The existence of the ``climatology''
  attribute of a CF-netCDF coordinate variable corresponds to a cell
  extent property describing that climatological time axis rules are
  in use.
  
\item Geometry. Each cell is a spatial representation of a real-world
  feature described by one or more sets of points, lines or
  polygons. The ``geometry\_type'' attribute of a CF-netCDF geometry
  container variable corresponds to a cell extent property describing
  the type of geometry in use. The ``part\_node\_count'' and
  ``interior\_ring'' attributes of the geometry container variable
  correspond to cell extent arrays for reconstructing the cell
  extents. These attributes contain references to other CF-netCDF
  variables, so the intra-cell connectivity arrays are constructed
  from the referenced variables' data.

\item Mesh topology. When a cell represents a three-dimensional volume
  enclosed by a set of faces, the ``volume\_shape\_type'' attribute of
  a CF-netCDF mesh topology variable corresponds to a cell extent
  array for reconstructing the cell extents. This attribute contains a
  reference to another CF-netCDF variable, so the cell extent array is
  constructed from the referenced variable's data.
  
\end{itemize}

%%  INTRA%%%%%%%%%%%%%%%%%

A coordinate construct may optionally contain one or more properties
to describe ``intra-cell connectivity'', i.e. how the vertices stored
in the cell bounds array relate to one another to describe cell
extents. These properties may be strings describing the relationship
between vertices and cell regions; or arrays providing ancillary
values for each cell to be used when recreating cell regions from the
given vertices. Such an intra-cell connectivity array spans the same
domain axes as its coordinate array, with the addition of an extra
dimension whose size is that of the maximum number of ancillary values
required for any cell (padding any unused elements with missing
values).

Intra-cell connectivity properties and arrays are created from
particular attributes of a variety of CF-netCDF variables; which ones
depends on the type of relationship between the cell bounds and the
cell regions:

\begin{itemize}

\item Climatological time axis. Each cell may be non-contiguous in
  time and the two vertices provided for each cell record the start
  and end of the whole period, wit the bounds for the cell's
  sub-regions being inferred from rules given in the CF
  conventions. The existence of the ``climatology'' attribute of a
  CF-netCDF coordinate variable corresponds to an intra-cell
  connectivity property specifying that climatological time axis rules
  are in use.
  
\item Geometry. Each cell is a spatial representation of a real-world
  feature described by one or more sets of points, lines or
  polygons. The ``geometry\_type'' attribute of a CF-netCDF geometry
  container variable corresponds to an intra-cell connectivity
  property specifying the type of geometry in use. The
  ``part\_node\_count'' and ``interior\_ring'' attributes of the
  geometry container variable correspond to intra-cell connectivity
  arrays requireed to reconstruct the cell region. These attributes
  contain references to other CF-netCDF variables, so the intra-cell
  connectivity arrays are constructed from the referenced variables'
  data.

\item Mesh topology. A cell could represent a three-dimensional volume
  enclosed by a set of faces. The ``volume\_shape\_type'' attribute of
  a CF-netCDF mesh topology variable corresponds to an intra-cell
  connectivity array. This attribute contains a reference to another
  CF-netCDF variable, so the intra-cell connectivity array is
  constructed from the referenced variable's data.
  
\end{itemize}

%%  INTER%%%%%%%%%%%%%%%%%

An auxiliary coordinate construct may optionally contain one or more
properties to describe ``inter-cell connectivity'', i.e. how cells
relate to each other (e.g. which nodes are connected in a
one-dimensionsal mesh topology) and to the cells of other, related
domains (e.g.\ which edges are connected ). These properties may be
strings describing the role of the cells being connected
(e.g. ``node'' in a mesh topology); or arrays providing ancillary
values that indicate relative position of all cells. Such an
inter-cell connectivity array need not span the same domain axes as
its coordinate array, with the addition of an extra dimension whose
size is that of the maximum number of ancillary values required for
any cell (padding any unused elements with missing values).




The extent of each cell is implied by the vertices stored in the cell
bounds array. By default it is assumed that an edge of a cell is
defined by adjacent vertices in the cell bounds array, as well as an
edge between the last and first vertices, ignoring any missing
values. When this interpretation is not applicable, additional
``intra-cell connectivity'' properties and arrays may be used to
define the exact regions of the domain spanned by the cells. These
optional properties describe the type of relationship between vertices
and cell regions; and optional arrays may provide ancillary values to
be used when recreating the cell regions from the given vertices.

Connectivity properties and arrays are created from particular
attributes of particular CF-netCDF variables; which ones depends on
the type of relationship between the cell bounds and the cell regions:

%%%%%%%%%%%%%%%%%

The extent of each cell is implied by the vertices stored in the cell
bounds array. By default it is assumed that an edge of a cell is
defined by two adjacent vertices in the cell bounds array, as well as
an edge between the last and first vertices, and ignoring any missing
values. When this interpretation is not applicable, additional
``connectivity'' properties and arrays may be used to define the exact
regions of the domain spanned by the cells. These optional properties
describe the type of relationship between vertices and cell regions;
and optional arrays may provide ancillary values to be used when
recreating the cell regions from the given vertices.

Connectivity properties and arrays are created from particular
attributes of particular CF-netCDF variables; which ones depends on
the type of relationship between the cell bounds and the cell regions:

\begin{itemize}

\item Climatological time axis. Each cell may be non-contiguous in
  time and the two vertices for each cell record the start and end of
  the whole period. The presence of the ``climatology'' attribute of a
  CF-netCDF coordinate variable indicates that a climatological time
  axis is in use, and a connectivity property should record this fact.
  
\item Geometry. Each cell is a spatial representation of a real-world
  feature described by points, lines or polygons; each cell having an
  arbitrary number of vertices. A connectivity property recording that
  the cells are defined by geometries, and the type of geometry, is
  created from the ``geometry\_type'' attribute of a CF-netCDF
  geometry container variable. Connectivity arrays may be required to
  define the vertices that apply to different sub-regions of the
  cells. These are derived from ``part\_node\_count'' and
  ``interior\_ring'' attributes of the geometry container
  variable. These attributes contain references to other CF-netCDF
  variables, so the connectivity arrays take the variable arrays as
  their values. It is not necessary to create a connectivity variable
  from the ``node\_count'' attribute, as this is as an aspect of
  unpacking the data from its space-saving ragged presentation.

\item Domain topology. In general, an auxiliary coordinate construct
  that corresponds to a topology coordinate construct is sufficiently
  self-describing without any ancillary properties. When a cell is a
  three-dimensional volume enclosed by a set of faces, however, a
  connectivity array must be created from the ``volume\_shape\_type''
  attribute of a CF-netCDF mesh variable. This contains a reference to
  another CF-netCDF variable so the connectivity array is that
  variable's data array.
  
\end{itemize}



By default it is assumed that each celasll descibes a contiguous region
for which adjacent vertices in the cell bounds arrays are assumed to
be adjacent vertices in the polygon whose adjacent vertices are also adjacent in the
cell bounbds array, with the last vertex also being connected to the
first and ignoring any missing values.

But in some cases a cell vertex may be connected with fewer or more
than two other vertices---which can occur if the cell describes a
geometrical point, line, volume, or higher-dimensional region. In
these cases the coordinate construct may require ancillary properties
or arrays to provide the connectivity bwetween

a single cell may be a collection of sub-regions,
each of which is described by an independent subset of the cell bounds
values; or a cell vertex may be connected with fewer or more than two
other vertices---which can occur if the cell describes a geometrical
point, line, volume, or higher-dimensional region. In these cases the
coordinate construct may require ancillary properties or arrays to
provide the connectivity bwetween



to describe a cell that encloses a contiguous
region. But in some cases a single cell may be a collection of
sub-cells, each of which is described by a subset of the cell bounds
values; or a cell vertex may be connected with fewer or more than two
other vertices---which can occur if the cell describes a


By default it is assumed that the vertices described by the cell
bounds array are coinnected to their adjacent vertices in that array
(ignoring any that are missing values) to describe a cell that
encloses a contiguous region. But in some cases a single cell may be a
collection of sub-cells, each of which is described by a subset of the
cell bounds values; or a cell vertex may be connected with fewer or
more than two other vertices---which can occur if the cell describes a
geometrical point, line, volume, or higher-dimensional region. In
these cases the coordinate construct may require ancillary properties
or arrays to provide the connectivity bwetween 



required to correctly interpret the nature of
its cells when this is not possible from the other properties,
coordinate values and cell bounds (if present). For example,
additional information may be required to reconstruct the exact extent
of a cell that spans a non-contiguous region. Ancillary properties
correspond to netCDF attributes of variables that contain the
necessary information. In general, an ancillary property will have the
same value as its netCDF attribute counterpart. If the value of a
netCDF attribute is a pointer to another netCDF variable then,
depending on the context, the ancillary property could be set to a
boolean (indicating that a framework defined alsewhere should be
applied), or the referenced variable's array of values (including its
units). In the latter case it is assumed that the array contains
information for describing each cell, and so spans the same domain
axes as the coordinate array, with the addition of any number
(including zero) of extra dimensions.

%A coordinate construct may also have optional ancillary properties for
%providing information required to correctly interpret the nature of
%its cells when this is not possible from the other properties,
%coordinate values and cell bounds (if present). For example,
%additional information may be required to reconstruct the exact extent
%of a cell that spans a non-contiguous region. Ancillary properties
%correspond to netCDF attributes of variables that contain the
%necessary information. In general, an ancillary property will have the
%same value as its netCDF attribute counterpart. If the value of a
%netCDF attribute is a pointer to another netCDF variable then,
%depending on the context, the ancillary property could be set to a
%boolean (indicating that a framework defined alsewhere should be
%applied), or the referenced variable's array of values (including its
%units). In the latter case it is assumed that the array contains
%information for describing each cell, and so spans the same domain
%axes as the coordinate array, with the addition of any number
%(including zero) of extra dimensions.

Ancillary properties are required in the following cases:

\begin{itemize}

\item Climatological time axis. Each cell is non-contiguous in time
  and the cell bounds record the start and end of the whole
  period. The presence of a ``climatology'' attribute of a CF-netCDF
  coordinate variable is used to create an ancillary property with the
  boolean true value. In CF-netCDF, the climatolgy attribute actually
  references the coordinate's bounds variable but, as these are
  already contained in the coordinate construct, it is sufficient to
  note that a climatalogical time axis is in use.
  
\item Geometry. Each cell is a spatial representation of a real-world
  feature described by points, lines or polygons; each cell having an
  arbitrary number of vertices. The vertex locations, stored in the
  cell bounds array, are defined by a CF-netCDF variable referenced by
  the ``node\_coordinates'' attribute of a CF-netCDF geometry
  container variable. Ancillary properties are created from the
  ``geometry\_type'', ``part\_node\_count'' and ``interior\_ring''
  attributes of the geometry container variable. The latter two
  attributes contain references to other CF-netCDF variables and the
  corresponding ancillary properties are represented by the variables'
  data arrays. It is not necessary to create an ancillary variable
  from the ``node\_count'' attribute, as this is as an aspect of
  unpacking the data from its space-saving ragged presentation.

\item Domain topology. In general, an auxiliary coordinate construct
  that corresponds to a topolgy coordinate construct is sufficiently
  self-describing without any ancillary properties. When a cell is a
  three-dimensional volume enclosed by a set of faces, however, an
  ancillary property must be created from the ``volume\_shape\_type''
  attribute of a CF-netCDF mesh variable. This contains a reference to
  another CF-netCDF variable and the corresponding ancillary property
  is represented by the variable's data array.
  
\end{itemize}

Domain topology construct

<preamble>

The domain topology construct consists of:

* Properties (e.g. mesh type; mesh name; an integer recording the
  highest dimensionality of the geometric elements)

* Topology dimensions [1..*] numbers of nodes, edges, faces, volumes in the
  domain topology). Nodes are mandatory.

* Topology coordinate constructs [1..*] ((representative) locations
  of nodes, edges, faces, volumes). Node-? connectivity mandatory

* Topology connectivity constructs [1..*] (indices connecting topology elements)


