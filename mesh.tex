4.3 Coordinates: dimension coordinate and auxiliary constructs

Coordinate constructs (Fig. 10) provide information which locate the
cells of the domain and which depend on a subset of the domain axis
constructs. As previously discussed, there are two distinct types of
coordinate construct: a dimension coordinate construct provides
monotonic numeric coordinates for a single domain axis, and an
auxiliary coordinate construct provides any type of coordinate
information for one or more of the domain axes.

In both cases, the coordinate construct consists of a data array of
the coordinate values which spans a subset of the domain axis
constructs, an optional array of cell bounds recording the extents of
each cell, and properties to describe the coordinates (in the same
sense as for the field construct). An array of cell bounds spans the
same domain axes as its coordinate array, with the addition of an
extra dimension whose size is that of the number of vertices of each
cell. This extra dimension does not correspond to a domain axis
constructsince it does not relate to an independent axis of the domain
(for example, the bounds dimension defined at line 6 in Fig. 3 does
not correspond to a domain axis construct).

The dimension coordinate construct is able to unambiguously describe
cell locations because a domain axis can be associated with at most
one dimension coordinate construct, whose data array values must all
be non-missing and strictly monotonically increasing or
decreasing. They must also all be of the same numeric data type. If
cell bounds are provided, then each cell must have exactly two
vertices. CF-netCDF coordinate variables and numeric scalar coordinate
variables correspond to dimension coordinate constructs.

Auxiliary coordinate constructs have to be used, instead of dimension
coordinate constructs, when a single domain axis requires more then
one set of coordinate values, when coordinate values are not numeric,
strictly monotonic, or contain missing values, or when they vary along
more than one domain axis construct simultaneously. CF-netCDF
auxiliary coordinate variables and non-numeric scalar coordinate
variables correspond to auxiliary coordinate constructs.

If a domain axis construct does not correspond to a continuous
physical quantity, then it is not necessary for it to be associated
with a dimension coordinate construct. For example, this is the case
for an axis that runs over ocean basins or area types, or for a domain
axis that indexes a time series at scattered points. In such cases,
one-dimensional auxiliary coordinate constructs could be used to store
coordinate values. These axes are discrete axes in CF-netCDF.

A coordinate construct may also have optional ancillary properties for
providing information required to correctly interpret the nature of
its cells when this is not possible from the other properties,
coordinate values and cell bounds (if present). For example,
additional information may be required to reconstruct the exact extent
of a cell that spans a non-contiguous region. Ancillary properties
correspond to netCDF attributes of variables that contain the
necessary information. In general, an ancillary property will have the
same value as its netCDF attribute counterpart. If the value of a
netCDF attribute is a pointer to another netCDF variable then,
depending on the context, the ancillary property could be set to a
boolean (indicating that a framework defined alsewhere should be
applied), or the referenced variable's array of values (including its
units). In the latter case it is assumed that the array contains
information for describing each cell, and so spans the same domain
axes as the coordinate array, with the addition of any number
(including zero) of extra dimensions.

Ancillary properties are required in the following cases:

\begin{itemize}

\item Climatological time axis. Each cell is non-contiguous in time
  and the cell bounds record the start and end of the whole
  period. The presence of a ``climatology'' attribute of a CF-netCDF
  coordinate variable is used to create an ancillary property with the
  boolean true value. In CF-netCDF, the climatolgy attribute actually
  references the coordinate's bounds variable but, as these are
  already contained in the coordinate construct, it is sufficient to
  note that a climatalogical time axis is in use.
  
\item Geometry. Each cell is a spatial representation of a real-world
  feature described by points, lines or polygons; each cell having an
  arbitrary number of vertices. The vertex locations, stored in the
  cell bounds array, are defined by a CF-netCDF variable referenced by
  the ``node\_coordinates'' attribute of a CF-netCDF geometry
  container variable. Ancillary properties are created from the
  ``geometry\_type'', ``part\_node\_count'' and ``interior\_ring''
  attributes of the geometry container variable. The latter two
  attributes contain references to other CF-netCDF variables and the
  corresponding ancillary properties are represented by the variables'
  data arrays. It is not necessary to create an ancillary variable
  from the ``node\_count'' attribute, as this is as an aspect of
  unpacking the data from its space-saving ragged presentation.

\item Domain topology. In general, an auxiliary coordinate construct
  that corresponds to a topolgy coordinate construct is sufficiently
  self-describing without any ancillary properties. When a cell is a
  three-dimensional volume enclosed by a set of faces, however, an
  ancillary property must be created from the ``volume\_shape\_type''
  attribute of a CF-netCDF mesh variable. This contains a reference to
  another CF-netCDF variable and the corresponding ancillary property
  is represented by the variable's data array.
  
\end{itemize}

Domain topology construct

<preamble>

The domain topology construct consists of:

* Properties (e.g. mesh type; mesh name; an integer recording the
  highest dimensionality of the geometric elements)

* Topology dimensions [1..*] numbers of nodes, edges, faces, volumes in the
  domain topology). Nodes are mandatory.

* Topology coordinate constructs [1..*] ((representative) locations
  of nodes, edges, faces, volumes). Node-? connectivity mandatory

* Topology connectivity constructs [1..*] (indices connecting topology elements)


